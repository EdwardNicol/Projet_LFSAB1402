\documentclass[a4paper,12pt,oneside ]{article}

\usepackage[utf8]{inputenc}
\usepackage[T1]{fontenc}      
\usepackage[francais]{babel}

\usepackage{graphicx}

\title{LFSAB1403 : DJ'Oz}
\author{Edward \bsc{Nicol} (27101300) \\ Virgile \bsc{Goyens}(83391300)}
\date{\today}

\begin{document}
\maketitle

\begin{figure}[h]
\begin{center}
	\includegraphics[scale=0.3]{DJOZ}
\end{center}
\end{figure}

\newpage
\section{Structure du programme}
	Le programme est divisé en deux grandes fonctions: la fonction \texttt{fun \{Interprete Partition\}} et la fonction \texttt{fun \{Mix Interprete Music\}}.
	
	Traitons dans un premier temps la fonction \texttt{fun \{Interprete Partition\}}. Cette fonction prend une partition comme argument et renvoie une voix, c'est à dire une liste d'échantillons. Cette fonction possède trois fonction locales: \texttt{fun\{ToNote Note\}}, \texttt{fun\{CountNotes Partition Acc\}}, \texttt{fun\{GetEchantillon Note Facteur Transposer\}} et fun \texttt{\{SuperInterprete Partition Bourdon Facteur Transposer\}}.
	
	Les 3 premières sont des fonctions utilisées dans le programme. \texttt{\{ToNote\}} permet d'uniformiser le format des notes. \texttt{\{CountNotes\}} permet de compter le nombre de notes d'une partition (utile dans le cas d'une transformation du type \texttt{duree()}). Finalement, \texttt{\{GetEchantillon\}} renvoie un échantillon d'une note quelconque, en prenant en compte plusieurs paramètres.
	
	La fonction \texttt{\{SuperInterprete\}} quant à elle représente le corps de la fonction. Grâce à ses paramètres supplémentaires, elle permet de tenir compte des modifications à apporter lors de la création d'un échantillon (du point de vue de la durée et/ou de la hauteur de la note. Elle utilise le principe du pattern matching pour interpréter la liste partition. Premièrement, un appel à la fonction \texttt{\{Flatten\}} est effectué pour ne plus avoir de liste imbriquées. Ensuite, pour chaque élément de la partition que la fonction rencontrera, elle vérifiera s'il s'agit d'une modification: le cas échéant, elle effectuera une modification dans les paramètres de la fonction \texttt{\{SuperInterprete\}}. Sinon, la fonction considérera que l'élément de partition est une note: elle fera donc appel à la fonction \texttt{GetEchantillon} et un appel récursif sera fait sur la queue de la liste. Le cas ou la partition ne comporte qu'un élément est aussi pris en compte.
\end{document}